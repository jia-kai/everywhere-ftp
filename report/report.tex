% $File: report.tex
% $Date: Wed Jan 01 18:26:19 2014 +0800
% $Author: jiakai <jia.kai66@gmail.com>

\documentclass[a4paper]{article}
\usepackage{amsmath,amssymb,amsthm,fontspec,zhspacing,verbatim,graphicx}
\usepackage[hyperfootnotes=false,colorlinks,linkcolor=black,anchorcolor=black,citecolor=black]{hyperref}
\usepackage[top=2in, bottom=1.5in, left=1in, right=1in]{geometry}

\newcommand{\addplot}[1]{\begin{center}
	\includegraphics[width=0.8\textwidth]{#1}
\end{center}}

% \textref{marker}{text}
\newcommand{\textref}[2]{\hyperref[#1]{#2}}
\newcommand{\figref}[1]{\hyperref[fig:#1]{Figure~\ref*{fig:#1}}}
\newcommand{\eqnref}[1]{\hyperref[eqn:#1]{(\ref*{eqn:#1})}}
\newcommand{\lemref}[1]{\hyperref[lemma:#1]{Lemma~\ref*{lemma:#1}}}


\zhspacing

\title{FTP实验报告}
\author{清华大学\\计14~贾开\\<jia.kai66@gmail.com>}
\date{\today}

\begin{document}
\maketitle
\tableofcontents

\section{实验目的}
遵循RFC959及其它相关标准,在Linux上实现FTP客户端及服务器,
进一步了解TCP/IP协议以及socket编程。

\section{实验过程概述}
本次实验在linux系统下完成,使用了C++11作为开发语言。
首先将系统的socket API封装成了C++类,方便调用,且提供更安全的自动资源管理
和更完善的异常处理。

在服务器端,使用分离的线程处理每一个客户连接,在一个主循环中不断读取客户命令并进
行相应处理。目前实现了passive模式以及以下命令: 
\begin{quote}
	ALLO, CWD, DELE, FEAT, LIST, MKD, NLST, PASS, PASV, PWD, QUIT, RETR, RMD,
	SIZE, STOR, SYST, TYPE, USER
\end{quote}

服务器能够支持新建、删除目录,上传、下载、删除文件,列出目录内容,改变当前目录。
可以使用lftp, firefox, chrome等常见客户端访问,运行效果如\figref{server}所示,
其中左列是服务器日志信息,右列是lftp访问的操作过程。
\begin{figure}[h!]
	\addplot{../img/server.png}
	\caption{\label{fig:server}服务器运行效果截图}
\end{figure}

客户端能够进行与服务器所实现的操作对应的操作,包括用匿名用户登录、
取当前工作目录、改变工作目录、取目录内容列表、新建目录、删除目录、
上传文件、下载文件、删除文件,客户端可以与标准ftp服务器(如vsftpd)通信。


\section{实验亮点}
\begin{enumerate}
	\item 遵循RFC959,服务器与客户端均可单独与第三方客户端、服务器对接。
	\item 使用C++11作为开发语言,兼顾语言表达的灵活性、系统API调用的便利性以及
		程序效率。其中的\verb|shared_ptr|, \verb|auto|, \verb|thread|等极大地提
		高了开发效率和程序鲁棒性。
	\item 使用独立线程而非进程处理每个客户连接,保障了并发性和效率。
	\item 安全性考虑,包括对每个用户文件操作先要检查是否在设定的FTP根目录内,
		以及对被读写的文件进行类型判断,只能是regular file,不能是管道、设备文件
		等系统中的特殊文件。
\end{enumerate}

\section{性能评测}
pyftpdlib(\url{http://code.google.com/p/pyftpdlib})
是一个开源的python ftp服务器实现,
并且自带了一套benchmark框架。pyftpdlib的性能不输C实现的vsftpd。
在此我也利用其框架测试了我的性能,并在相同环境下测试pyftpdlib性能以便对比。
其中,我实现的ftp服务器命名为wftp。

\subsection{测试环境}
\begin{center}\begin{tabular}{ll}
	CPU & Intel(R) Core(TM) i5-2450M CPU @ 2.50GHz, 2 cores, 2 threads/core \\
	RAM & 5.7G \\
	OS & arch linux \\
	kernel & linux 3.10.7, customized configuration \\
	gcc &   4.8.1 20130725 \\
	compiler optimization & -O2 \\
	python & 3.3.3 
\end{tabular}\end{center}

\subsection{测试结果}
\begin{center}\begin{tabular}{l|l|l|l}
	\hline
	benchmark type & pyftpdlib 1.3.0 & wftp & speedup \\ \hline
	STOR (client -> server) & 689.72 MB/sec & 770.09 MB/sec & 11.65\% \\
	RETR (server -> client) & 1517.02 MB/sec & 1564.69 MB/sec & 3.14\% \\
	300 concurrent clients (connect, login) & 2.26 secs & 0.83 secs & 172.3\% \\
	STOR (1 file with 300 idle clients) & 
		688.93 MB/sec & 906.02 MB/sec & 31.51\% \\
	RETR (1 file with 300 idle clients) &
		1534.30 MB/sec & 1465.18 MB/sec & -4.50\% \\
	300 concurrent clients (RETR 10.0M file) &
		3.30 secs & 4.97 secs & -33.60\% \\
	300 concurrent clients (STOR 10.0M file) &
		8.11 secs & 8.24 secs & -1.58\% \\
	300 concurrent clients (QUIT) &  0.07 secs & 0.01 secs & 700\% \\
	\hline
\end{tabular}\end{center}

其中,speedup 一列的计算方法为:
\begin{quote}
	对于每项评测内容,设pyftpdlib的值为$a$,wftp的值为$b$。
	\begin{itemize}
		\item 如果$a$, $b$的单位是MB/sec,则
			\[ speedup = \dfrac{b-a}{a} \]
		\item 如果$a$, $b$的单位是secs,则
			\[ speedup = \dfrac{b^{-1}-a^{-1}}{a^{-1}} = \dfrac{a-b}{b} \]
	\end{itemize}
\end{quote}

由评测结果可以看出,wftp作为一个轻量级C++实现,
其在连续I/O下还是有明显性能优势的;
但由于采用了线程,而pyftpdlib使用的是异步IO,故在高并发下性能明显减弱。

\section{实验思考题}
\begin{enumerate}
	\item 为何需要建立命令和数据两个连接? \\
		这样可以在数据传输的同时执行非数据命令,还可以利用主动、被动方式直接在服
		务器间传输数据无需中转。
	\item 关于主动方式和被动方式 \\
		两者均用于建立数据连接,主动方式是服务器连接到客户端,被动方式是客户端连
		接到服务器。一般而言仅需被动方式即可,这样的好处是,可以通过客户端控制两
		台服务器分别以主动和被动方式建立连接,让两台服务器间传输数据无需客户端中
		转。
	\item 大量小文件传输 \\
		这是因为FTP协议中每传输一个文件都需要发送命令,等待服务器回应,并重新建
		立数据连接。可以将文件打包传输。
\end{enumerate}

\end{document}

% vim: filetype=tex foldmethod=marker foldmarker=f{{{,f}}}

